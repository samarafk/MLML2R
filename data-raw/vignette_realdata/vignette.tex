\documentclass{article}
\RequirePackage{/Library/Frameworks/R.framework/Versions/3.4/Resources/library/BiocStyle/resources/tex/Bioconductor}

\AtBeginDocument{\bibliographystyle{/Library/Frameworks/R.framework/Versions/3.4/Resources/library/BiocStyle/resources/tex/unsrturl}}

\bioctitle[MLML2R]{MLML2R: Maximum Likelihood Estimates for 5-mC and 5-hmC Levels in the DNA}
\author{Samara F. Kiihl\thanks{\email{samara@ime.unicamp.br}}\, and Maria Tellez-Plaza}



\usepackage{Sweave}
\begin{document}

\maketitle

\begin{abstract}

We present a guide to the Bioconductor package \Biocpkg{MLML2R}. The package provides computational efficient maximum likelihood estimates of DNA methylation and hydroxymethylation proportions when data from the DNA processing methods bisulfite conversion (BS), oxidative bisulfite conversion (ox-BS), and Tet-assisted bisulfite conversion (TAB) are available. Estimates can be obtained when data from all the three methods are available or when any combination of only two of them are available. The package does not depend on other \R{} packages, allowing the user  to read and preprocess the data in any software and import the results into \R{} in matrix format, obtain the estimates and use that as input in the other packages for genomic analysis, such as \Biocpkg{minfi}, \Biocpkg{sva} and \Biocpkg{limma}.

\end{abstract}

Report issues on \url{https://github.com/samarafk/MLML2R/issues}


\tableofcontents

\newpage

\section{Introduction}

In a given CpG site from a single cell we will either have a $C$ or a $T$ after DNA processing conversion methods, with a different interpretation for each of the available methods. This is a binary outcome and we assume a Binomial model and use the maximum likelihood estimation method to obtain the estimates for hydroxymethylation and methylation proportions.

$T$ reads are referred to as converted cytosine and $C$ reads are referred to as unconverted cytosine. Conventionally, $T$ counts are also referred to as unmethylated counts, and $C$ counts as methylated counts. In case of Infinium Methylation arrays, we have intensities representing the methylated (M) and unmethylated (U) channels that are proportional to the number of unconverted and converted cytosines ($C$ and $T$, respectively). The most used summary from these experiments is  the proportion $\beta=\frac{M}{M+U}$, commonly referred to as \textit{beta-value}, which reflects the methylation level at a CpG site. Na\"ively using the difference between betas from BS and oxBS as an estimate of 5-hmC (hydroxymethylated cytosine), and the difference between betas from BS and TAB as an estimate of 5-mC (methylated cytosine) can many times provide negative proportions and instances where the sum of 5-C (unmodified cytosine), 5-mC and 5-hmC proportions is greater than one due to measurement errors.

\Biocpkg{MLML2R} package allows the user to jointly estimate hydroxymethylation and methylation consistently and efficiently.

The function \Rfunction{MLML} takes as input the data from the different methods and returns the estimated proportion of methylation, hydroxymethylation and unmethylation for a given CpG site. Table \ref{table:args} presents the arguments of the \Rfunction{MLML} and Table \ref{table:value} lists the results returned by the function.

The function assumes that the order of the rows and columns in the input matrices are consistent. In addition, all the input matrices must have the same dimension. Usually, rows represent CpG loci and columns are the samples.


\begin{table}[h]
\begin{center}
\begin{tabular}{|l|l|}
\hline
Arguments           &       Description \\
\hline
\Robject{G.matrix} & Unmethylated channel (Converted cytosines/ T counts) from\\
 & TAB-conversion (reflecting 5-C + 5-mC). \\
\Robject{H.matrix} & Methylated channel (Unconverted cytosines/ C counts) from\\
 & TAB-conversion (reflecting True 5-hmC).\\
\Robject{L.matrix}	& Unmethylated channel (Converted cytosines/ T counts) from\\
 & oxBS-conversion (reflecting 5-C + 5-hmC).\\
\Robject{M.matrix}	& Methylated channel (Unconverted cytosines/ C counts) from\\
 & oxBS-conversion (reflecting True 5-mC).\\
\Robject{T.matrix}	&
Methylated channel (Unconverted cytosines/ C counts) from\\
 & standard BS-conversion (reflecting 5-mC+5-hmC).\\
\Robject{U.matrix}	&
Unmethylated channel (Converted cytosines/ T counts) from\\
 & standard BS-conversion (reflecting True 5-C).\\
\Robject{iterative}	& logical. If iterative=TRUE EM-algorithm is used.\\
 & For the combination of two methods, iterative=FALSE returns the\\
 & exact constrained MLE using the the pool-adjacent-violators algorithm\\
 &  (PAVA). When all three methods are combined, iterative=FALSE\\
 &  returns the constrained MLE using Lagrange multiplier.\\
\Robject{tol}	&  convergence tolerance; considered only if iterative=TRUE.\\
\hline
\end{tabular}
\end{center}
\caption{Arguments of the \Rfunction{MLML} function}
\label{table:args}
\end{table}


	\begin{table}[h]
			\begin{center}
				\begin{tabular}{|l|l|}
					\hline
					Value           &       Description \\
					\hline
					\Robject{mC}	&
maximum likelihood estimate for the 5-mC proportion\\
		\Robject{hmC}	&
maximum likelihood estimate for the 5-hmC proportion\\
\Robject{C}	&
maximum likelihood estimate for the 5-mC proportion\\
\Robject{methods}	&
the conversion methods used to produce the MLE\\
				 \hline
				\end{tabular}
			\end{center}
			\caption{Results returned from the \Rfunction{MLML} function}
			\label{table:value}
		\end{table}



\section{Worked examples}

When only two methods are available, the default option returns the exact constrained maximum likelihood estimates  using the the pool-adjacent-violators algorithm (PAVA)\cite{ayer1955}. Maximum likelihood estimate via EM-algorithm approach \cite{Qu:MLML} is obtained with the option \verb|iterative=TRUE|. In this case, the default (or user specified) \verb|tol| is considered in the iterative method.

When data from the three methods are available, the default otion in the \Rfunction{MLML} function returns the constrained maximum likelihood estimates using an approximated solution for the Lagrange multipliers method. Maximum likelihood estimate via EM-algorithm approach \cite{Qu:MLML} is obtained with the option \verb|iterative=TRUE|. In this case, the default (or user specified) \verb|tol| is considered in the iterative method.

\Biocpkg{MLML2R} includes small example datasets for illustration and basic function calling examples for all the different cases in the help of the function \Rfunction{MLML}.


Here we show examples from real datasets.


\subsection{Publicly available data: GSE63179}

We will use the dataset from \cite{10.1371/journal.pone.0118202}, which consists of eight DNA samples from the same DNA source treated with oxBS-BS and hybridized to the Infinium 450K array.

When data is obtained through Infinium Methylation arrays, we recommend the use of the \Biocpkg{minfi} package, a well-established tool for reading, preprocessing and analysing DNA methylation data from these platforms. Although our example relies on \Biocpkg{minfi} and other \Bioconductor{} tools, \Biocpkg{MLML2R} does not depend on any packages. Thus, the user is free to read and preprocess the data using any software of preference and then import the intensities (or $T$ and $C$ counts) for the methylated and unmethylated channel (or converted and uncoverted cytosines) into \R{} in matrix format.

To start this example we will need the following packages:
\begin{Schunk}
\begin{Sinput}
> library(MLML2R)
> library(minfi)
> library(GEOquery)
\end{Sinput}
\end{Schunk}


It is usually best practice to start the analysis from the raw data, which in the case of the 450K array is a \verb|.IDAT| file.

The raw files are deposited in GEO and can be downloaded by using the \Rfunction{getGEOSuppFiles}. There are two files for each replicate, since the 450k array is a two-color array. The \verb|.IDAT| files are downloaded in compressed format and need to be uncompressed before they are read by the \Rfunction{read.metharray.exp} function.

\begin{Schunk}
\begin{Sinput}
> getGEOSuppFiles("GSE63179")
> untar("GSE63179/GSE63179_RAW.tar", exdir = "GSE63179/idat")
> list.files("GSE63179/idat", pattern = "idat")
> files <- list.files("GSE63179/idat", pattern = "idat.gz$", full = TRUE)
> sapply(files, gunzip, overwrite = TRUE)
\end{Sinput}
\end{Schunk}

The \verb|.IDAT| files can now be read:
\begin{Schunk}
\begin{Sinput}
> rgSet <- read.metharray.exp("GSE63179/idat")
\end{Sinput}
\end{Schunk}

To access phenotype data we use the \Rfunction{pData} function. The phenotype data is not yet available from the \Robject{rgSet}.
\begin{Schunk}
\begin{Sinput}
> pData(rgSet)
\end{Sinput}
\begin{Soutput}
DataFrame with 8 rows and 0 columns
\end{Soutput}
\end{Schunk}
In this example the phenotype is not really relevant, since we have only one sample: male, 25 years old. What we do need is the information about the conversion method used in each replicate: BS or oxBS. We will access this information automatically from GEO:
\begin{Schunk}
\begin{Sinput}
> geoMat <- getGEO("GSE63179")
> pD.all <- pData(geoMat[[1]])
> pD <- pD.all[, c("title", "geo_accession", "characteristics_ch1.1",
+                  "characteristics_ch1.2","characteristics_ch1.3")]
> pD
\end{Sinput}
\end{Schunk}


\begin{Schunk}
\begin{Soutput}
                  title geo_accession characteristics_ch1.1 characteristics_ch1.2
GSM1543269   brain-BS-1    GSM1543269          gender: Male               age: 25
GSM1543270 brain-oxBS-3    GSM1543270          gender: Male               age: 25
GSM1543271   brain-BS-2    GSM1543271          gender: Male               age: 25
GSM1543272 brain-oxBS-4    GSM1543272          gender: Male               age: 25
GSM1543273   brain-BS-3    GSM1543273          gender: Male               age: 25
GSM1543274   brain-BS-4    GSM1543274          gender: Male               age: 25
GSM1543275 brain-oxBS-1    GSM1543275          gender: Male               age: 25
GSM1543276 brain-oxBS-2    GSM1543276          gender: Male               age: 25
           characteristics_ch1.3
GSM1543269    bisulfite_proc: BS
GSM1543270  bisulfite_proc: oxBS
GSM1543271    bisulfite_proc: BS
GSM1543272  bisulfite_proc: oxBS
GSM1543273    bisulfite_proc: BS
GSM1543274    bisulfite_proc: BS
GSM1543275  bisulfite_proc: oxBS
GSM1543276  bisulfite_proc: oxBS
\end{Soutput}
\end{Schunk}


This phenotype data needs to be merged into the methylation data. The following commands guarantee we have the same replicate identifier in both datasets before merging.

\begin{Schunk}
\begin{Sinput}
> sampleNames(rgSet) <- sapply(sampleNames(rgSet),function(x)
+   strsplit(x,"_")[[1]][1])
> rownames(pD) <- pD$geo_accession
> pD <- pD[sampleNames(rgSet),]
> pData(rgSet) <- as(pD,"DataFrame")
> rgSet
\end{Sinput}
\begin{Soutput}
class: RGChannelSet 
dim: 622399 8 
metadata(0):
assays(2): Green Red
rownames(622399): 10600313 10600322 ... 74810490 74810492
rowData names(0):
colnames(8): GSM1543269 GSM1543270 ... GSM1543275 GSM1543276
colData names(5): title geo_accession characteristics_ch1.1
  characteristics_ch1.2 characteristics_ch1.3
Annotation
  array: IlluminaHumanMethylation450k
  annotation: ilmn12.hg19
\end{Soutput}
\end{Schunk}

The \Robject{rgSet} object is a class called \Rclass{RGChannelSet} used for two color data (green and a red channel). The input in the \Rfunction{MLML} funcion is  \Rclass{MethylSet}, which contains the methylated and unmethylated signals. The most basic way to construct a \Rclass{MethylSet} is  using the function \Rfunction{preprocessRaw}. Here we chose the function \Rfunction{preprocessNoob} for background correction and construction of the \Rclass{MethylSet}.


\begin{Schunk}
\begin{Sinput}
> MSet.noob<- preprocessNoob(rgSet)
\end{Sinput}
\begin{Soutput}
[preprocessNoob] Applying R/G ratio flip to fix dye bias...
\end{Soutput}
\end{Schunk}


After the preprocessed steps we can use \Rfunction{MLML} from the \Biocpkg{MLML2R} package.


The BS replicates are in columns 1, 3, 5, and 6. The remaining columns are from the oxBS treated replicates.

\begin{Schunk}
\begin{Sinput}
> MethylatedBS <- getMeth(MSet.noob)[,c(1,3,5,6)]
> UnMethylatedBS <- getUnmeth(MSet.noob)[,c(1,3,5,6)]
> MethylatedOxBS <- getMeth(MSet.noob)[,c(7,8,2,4)]
> UnMethylatedOxBS <- getUnmeth(MSet.noob)[,c(7,8,2,4)]
\end{Sinput}
\end{Schunk}


In this example we only have two methods, therefore we can choose between the EM-algorithm and the exact constrained maximum likelihood estimates (using PAVA).

Estimates via the EM-algorithm:

\begin{Schunk}
\begin{Sinput}
> results_emPD1 <- MLML(T.matrix = MethylatedBS , U.matrix = UnMethylatedBS,
+                    L.matrix = UnMethylatedOxBS, M.matrix = MethylatedOxBS,
+                    iterative = TRUE)
\end{Sinput}
\end{Schunk}


The exact constrained MLE (using PAVA):

\begin{Schunk}
\begin{Sinput}
> results_exactPD1 <- MLML(T.matrix = MethylatedBS , U.matrix = UnMethylatedBS,
+                       L.matrix = UnMethylatedOxBS, M.matrix = MethylatedOxBS)
\end{Sinput}
\end{Schunk}

The estimates are very similar for both methods:
\begin{Schunk}
\begin{Sinput}
> all.equal(results_exactPD1$hmC,results_emPD1$hmC,scale=1)
\end{Sinput}
\end{Schunk}

When only two methods are available, we highly recommend the default option \Rcode{iterative=FALSE} since the difference in the estimates obtained via EM and exact constrained is very small, but the former requires more computational effort.

% <<eval=FALSE,echo=TRUE,cache=TRUE>>=
% library(microbenchmark)
% mbmPD1 = microbenchmark(
%   EXACT = MLML(T.matrix = MethylatedBS , U.matrix = UnMethylatedBS,
%                L.matrix = UnMethylatedOxBS,M.matrix = MethylatedOxBS),
%   EM = MLML(T.matrix = MethylatedBS, U.matrix = UnMethylatedBS,
%             L.matrix = UnMethylatedOxBS, M.matrix = MethylatedOxBS,
%             iterative = TRUE),
%   times=1)
% mbmPD1
% @



% <<fig10,fig=TRUE,include=FALSE,width=15,height=5,eval=FALSE,echo=FALSE>>=
% par(mfrow =c(1,3))
% densityPlot(results_exactPD1$hmC,main= "5-hmC using PAVA",cex.axis=1.5,cex.main=1.5,cex.lab=1.5)
% densityPlot(results_exactPD1$mC,main= "5-mC using PAVA",cex.axis=1.5,cex.main=1.5,cex.lab=1.5)
% densityPlot(results_exactPD1$C,main= "5-C using PAVA",cex.axis=1.5,cex.main=1.5,cex.lab=1.5)
% @

\begin{figure*}[h]
 \includegraphics{vignette-fig10}
 \caption{\label{fig:fig10} Estimated proportions of hydroxymethylation, methylation and unmethylation for the CpGs in the dataset using the \Rfunction{MLML} function with default options.}
\end{figure*}

\subsection{Publicly available data: GSE73895}

We will use the dataset from \cite{pmid27886174}, which consists of 30 DNA samples from glioblastoma tumors treated with oxBS-BS and hybridized to the Infinium 450K array.


The raw files are deposited in GEO and can be downloaded and read into \R{} by doing:
\begin{Schunk}
\begin{Sinput}
> getGEOSuppFiles("GSE73895")
> untar("GSE73895/GSE73895_RAW.tar", exdir = "GSE73895/idat")
> idatFiles <- list.files("GSE73895/idat", pattern = "idat.gz$", full = TRUE)
> sapply(idatFiles, gunzip, overwrite = TRUE)
\end{Sinput}
\end{Schunk}

\begin{Schunk}
\begin{Sinput}
> rgSet <- read.metharray.exp("GSE73895/idat")
\end{Sinput}
\end{Schunk}

We need to identify the samples from different methods: BS-conversion, oxBS-conversion. We obtain this information from GEO:

\begin{Schunk}
\begin{Sinput}
> geoMat <- getGEO("GSE73895")
> pD.all <- pData(geoMat[[1]])
> pD <- pD.all[, c("title", "geo_accession", "characteristics_ch1",
+                  "characteristics_ch1.2","characteristics_ch1.3")]
> head(pD)
\end{Sinput}
\end{Schunk}


\begin{Schunk}
\begin{Soutput}
                   title geo_accession characteristics_ch1        characteristics_ch1.2
GSM1905306   tissue_1_BS    GSM1905306      gender: Female  survival time (months): 1.1
GSM1905307 tissue_1_oxBS    GSM1905307      gender: Female  survival time (months): 1.1
GSM1905308   tissue_2_BS    GSM1905308      gender: Female survival time (months): 53.3
GSM1905309 tissue_2_oxBS    GSM1905309      gender: Female survival time (months): 53.3
GSM1905310   tissue_3_BS    GSM1905310        gender: Male  survival time (months): 0.6
GSM1905311 tissue_3_oxBS    GSM1905311        gender: Male  survival time (months): 0.6
           characteristics_ch1.3
GSM1905306     subject age: 77.6
GSM1905307     subject age: 77.6
GSM1905308     subject age: 45.3
GSM1905309     subject age: 45.3
GSM1905310     subject age: 69.7
GSM1905311     subject age: 69.7
\end{Soutput}
\end{Schunk}

Keeping only some of the variables from phenotype data:
\begin{Schunk}
\begin{Sinput}
> names(pD)[c(1,3,4,5)] <- c("method","gender","survival_months","age_years")
> pD$gender <- sub("^gender: ", "", pD$gender)
> pD$age_years <- as.numeric(sub("^subject age: ", "", pD$age_years))
> pD$survival_months <- as.numeric(sapply(pD$survival_months, function(x)
+   strsplit(as.character(x),":")[[1]][2]))
> pD$method <- sapply(pD$method, function(x) strsplit(as.character(x),"_")[[1]][3])
\end{Sinput}
\end{Schunk}

We now need to merge this pheno data into the methylation data. The following are commands to make sure we have the same row identifier in both datasets before merging.

\begin{Schunk}
\begin{Sinput}
> sampleNames(rgSet) <- sapply(sampleNames(rgSet),function(x)
+   strsplit(x,"_")[[1]][1])
> rownames(pD) <- pD$geo_accession
> pD <- pD[sampleNames(rgSet),]
> pData(rgSet) <- as(pD,"DataFrame")
> rgSet
\end{Sinput}
\begin{Soutput}
class: RGChannelSet 
dim: 622399 60 
metadata(0):
assays(2): Green Red
rownames(622399): 10600313 10600322 ... 74810490 74810492
rowData names(0):
colnames(60): GSM1905306 GSM1905307 ... GSM1905364 GSM1905365
colData names(5): method geo_accession gender survival_months age_years
Annotation
  array: IlluminaHumanMethylation450k
  annotation: ilmn12.hg19
\end{Soutput}
\end{Schunk}

Preprocessing:
\begin{Schunk}
\begin{Sinput}
> MSet.noob<- preprocessNoob(rgSet)
\end{Sinput}
\begin{Soutput}
[preprocessNoob] Applying R/G ratio flip to fix dye bias...
\end{Soutput}
\end{Schunk}


After the preprocessed steps we can use \Rfunction{MLML} from the \Biocpkg{MLML2R} package.


\begin{Schunk}
\begin{Sinput}
> BS_index <- which(pData(rgSet)$method=="BS")
> oxBS_index <- which(pData(rgSet)$method=="oxBS")
> MethylatedBS <- getMeth(MSet.noob)[,BS_index]
> UnMethylatedBS <- getUnmeth(MSet.noob)[,BS_index]
> MethylatedOxBS <- getMeth(MSet.noob)[,oxBS_index]
> UnMethylatedOxBS <- getUnmeth(MSet.noob)[,oxBS_index]
\end{Sinput}
\end{Schunk}



In this example we only have two methods, therefore we can choose between the EM-algorithm and the exact constrained maximum likelihood estimates (using PAVA).

Estimates via the EM-algorithm:

\begin{Schunk}
\begin{Sinput}
> results_emPD2 <- MLML(T.matrix = MethylatedBS , U.matrix = UnMethylatedBS,
+                    L.matrix = UnMethylatedOxBS, M.matrix = MethylatedOxBS,
+                    iterative = TRUE)
\end{Sinput}
\end{Schunk}


The exact constrained MLE (using PAVA):

\begin{Schunk}
\begin{Sinput}
> results_exactPD2 <- MLML(T.matrix = MethylatedBS , U.matrix = UnMethylatedBS,
+                       L.matrix = UnMethylatedOxBS, M.matrix = MethylatedOxBS)
\end{Sinput}
\end{Schunk}

The estimates are very similar for both methods:
\begin{Schunk}
\begin{Sinput}
> all.equal(results_exactPD2$hmC,results_emPD2$hmC,scale=1)
\end{Sinput}
\end{Schunk}



\begin{figure*}[h]
 \includegraphics{vignette-fig11}
 \caption{\label{fig:fig11} Estimated proportions of hydroxymethylation, methylation and unmethylation for the CpGs in the dataset using the \Rfunction{MLML} function with default options.}
\end{figure*}


\bibliography{refs}


\end{document}
